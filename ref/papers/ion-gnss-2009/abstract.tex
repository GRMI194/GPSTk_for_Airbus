
\section*{Abstract}


The GPS Toolkit (GPSTk) project is an attempt to bring the power of
the open source world to the satellite navigation community.  The GPSTk is
intended to help eliminate the ``black box'' nature of many commercial
applications, and to enable rapid prototype development for data
analysis applications. It is also intended to support software development
and systems engineering associated with GNSS data collection systems.

Because satellite navigation is ubiquitous, its users employ
practically every computational architecture and operating system.
Therefore the GPSTk suite is intended to be as platform-independent as
possible.  This is achieved through use of ISO-standard C++.  The
principles of object-oriented programming are used throughout the
GPSTk code base in order to ensure that it is modular, extensible and
maintainable.

The software suite consists of a core library, auxiliary libraries,
and a large set of advanced applications. The libraries
provide the wide array of common functions that applications use to handle
data processing associated with GPS. Furthermore, programmers can also access
the library code to develop new processing applications.

The GPSTk was initially designed and developed in a highly
collaborative environment of software engineers and scientists in the
Space and Geophysics laboratory (SGL) at the Applied Research
Laboratories, The University of Texas at Austin (ARL:UT). SGL decided
in 2003 to open-source much of their basic GPS processing software as
the GPSTk under the GNU Lesser General Public License (LGPL) \mbox{version
2.1}.  The GPSTk's source code and documentation are now hosted on
public servers so that project members from multiple academic and
commercial institutions can freely collaborate in development.

In the last two years, a number of new applications and library
capabilities have been added to the GPSTk. New applications have been
added to characterize clock stability. New library capabilities have
been added to process GLONASS observations and to fully support the
Receiver INdependent EXchange (RINEX) version 3 file format. New
library code also provides the ability to generate highly customizable
plots compatible with \LaTeX\ and web browsers. A new library has been
added that provides a framework for precise point positioning (PPP).

